% Created 2022-10-21 Fri 18:49
% Intended LaTeX compiler: pdflatex
\documentclass[hidelinks,11pt]{article} \usepackage[utf8]{inputenc}
\usepackage[T1]{fontenc} \usepackage{graphicx} \usepackage{longtable}
\usepackage{wrapfig} \usepackage{rotating} \usepackage[normalem]{ulem}
\usepackage{amsmath} \usepackage{amssymb} \usepackage{capt-of}
\usepackage{hyperref} \usepackage{caption}
\usepackage{subcaption} % sufigures facilities
\usepackage{float} % for H option
\usepackage{xcolor} % for the textcolor command
\usepackage[a4paper,width=150mm,top=25mm,bottom=25mm]{geometry} % fixes margins
\usepackage{changepage}
\usepackage{float}
\author{Marcelo Veloso Maciel
  % \thanks{I thank Marek
  %   Kaminski, Donald Saari, and Ines Levin for their guidance and encouragement,
  %   and Maria Luiza Paschoal, Kaique Pereira Santos, John Brunson, Udita Ghosh
  %   and Joshua Storm for their feedback on this project.}
} \date{}

\title{Was Bolsonaro's 2018 electoral victory an institutional fluke?}


%Majoritarian principles at a critical juncture: an analysis of Brazil's 2018 presidential election

\hypersetup{
 pdfauthor={Marcelo Veloso Maciel},
 pdftitle={Majoritarian Principles},
 pdfkeywords={},
 pdfsubject={},
 pdfcreator={Emacs 28.2 (Org mode 9.6)},
 pdflang={English}}
\usepackage[authordate,strict,backend=biber,
bibencoding=inputenc]{biblatex-chicago}
%\setlength{\textfloatsep}{0.1cm}
\addbibresource{~/Main/Org/org-roam-mvm/bib/refs.bib}
\begin{document}

\maketitle
\begin{abstract}
  % Presidential Elections are critical moments for polyarchical systems,
  % particularly in contexts of high social tension. The 2018 presidential
  % election in Brazil used a two-round system, yet the most divisive candidates
  % went to the second round. Pairwise and positional voting procedures embody
  % different generalizationsq of a majoritarian credo that underpins such
  % elections. The paper mobilizes both perspectives and, using representative
  % survey data, reconstructs the top four preferences of the Brazilian electorate
  % a week before the election. It shows that the electoral winner, Jair Messias
  % Bolsonaro, was a Condorcet Winner, but may have not been the Borda Winner,
  % while the second-round loser, Fernando Haddad, was a Condorcet Loser.
  % Furthermore, possible alternative scenarios under different feasible sets of
  % candidates are simulated, contributing to understanding the role of decision
  % procedures in critical junctures.

  Presidential Elections are critical moments for polyarchical systems,
  particularly in contexts of high social tension. In this regard, the 2018
  presidential election in Brazil, which used a two-round system, serves as a
  significant case study. Intriguingly, the most divisive candidates went to the
  second round. Was this an institutional accident? Pairwise and positional
  voting procedures embody different generalizations of a majoritarian credo
  that underpins such elections. The paper mobilizes both perspectives and,
  using representative survey data, reconstructs the top four preferences of the
  Brazilian electorate a week before the election. The analysis reveals that the
  electoral winner, Jair Messias Bolsonaro, was a Condorcet Winner, but may have
  not been the Borda Winner. Conversely, the second-round loser, Fernando
  Haddad, was a Condorcet Loser. Furthermore, the paper explores possible
  alternative scenarios under different feasible sets of candidates through
  simulations, contributing to a deeper understanding of the role of decision
  procedures in critical junctures.


  \end{abstract}
% parencite for indirect citation, textcite for direct citation
\section{Introduction}

The escalating polarization, the looming threat of democratic backsliding, and a
broader concern for the longevity of democratic polities have sparked a
resurgence of interest in the role of institutions in either mitigating or
exacerbating destabilizing dynamics, and in the adaptability of the political
system in face of both internal and external stressors
\parencite{Bednare2113843118, chiopris2021wolf, ostrom1997meaning}. Given their
pivotal role in the input-output relationship between society and the state,
electoral institutions are naturally among the institutions under scrutiny
\parencite{Wange2021systems}. Consequently, a segment of the literature on
collective choice has speculated whether the recent electoral victories of
divisive candidates are a consequence of informationally deficient decision
procedures \parencite{potthoff2021condorcet, kurrild2018trump, woon2020trump}.

Jair Messias Bolsonaro is a hallmark example of a divisive candidate, and his
rise to power in the 2018 brazilian presidential election was marked by large
margins in his favor. Was his victory simply a byproduct of the decision
procedure? Despite the various historical contingencies that could explain his
triumph, it is natural to question the role of the electoral system in his
success. After all, it is widely recognized that the outcome of collective
choices is fundamentally dependent on the voting procedure
\parencite{riker1982liberalism}. Like Brazil's two-round system, most electoral
systems primarily use the first index of the voters' preferences
\parencite{grofman04_if_you_like_alter_vote}. This raises the question: how
would Bolsonaro have performed under informationally richer voting procedures,
such as methods based on pairwise comparisons and positional voting procedures?
Could he, and arguably other democratically elected destabilizing candidates, be
products of decision procedures that favor divisive candidates over more
inclusive ones \parencite{igersheim22_compar_votin_method}? What criteria should
be used to argue that a candidate's victory was an institutional artifact? Would
the result have changed with a different set of candidates?

The findings of this paper reveal that even though Bolsonaro was the Condorcet
winner, he would not have been victorious under all reasonable alternative
voting methods, and notably, he was not the Borda winner. His opponent in the
second round, Fernando Haddad, was a Condorcet loser among the top four
candidates and would not have defeated Bolsonaro under any positional voting
method. The third-place candidate, Ciro Gomes, had a slight positional advantage
over Bolsonaro, with the potential to beat him in \(53\%\) of the positional
voting methods. Ciro's advantage was especially pronounced in methods that
emphasized rejection. This case highlights a partial conflict between typical
evaluative positions, harking back to the historic Borda-Condorcet debate.

In order to arrive at these results, I first revisit the perspectives of Borda and Condorcet on voting procedures. Subsequently, I utilize a pre-electoral representative survey to reconstruct the full 4-top preferences - complete binary relations -  of the Brazilian population. This augmented data is then used to simulate electoral outcomes under alternative methods for the top 4 and 3 candidate sets. Finally, I discuss the significance of the results and conclude by delineating the limitations of this endeavor.

\section{Theory}

% The realization that the result of collective decisions is inseparable from the
% voting procedures being used and that such procedures differ in terms of their
% consistency with evaluative criteria naturally leads us to wonder what possible
% adjacent paths could have been taken in key electoral moments
% \parencite{tabarrok1999would, kaminski1999communism, ostrom1986agenda}.
% Concurrently, this realization puts us in a conundrum: what, among all possible
% criteria, should be used to motivate the counterfactual analysis? Isn't this
% endeavor inherently arbitrary, given that one can possibly retrofit a reasonable
% choice of voting method that matches an a-priori desired outcome
% \parencite{riker1982liberalism}? The anchoring point is to note that political players themselves reflect on those procedural properties, which end up being
% levers for their legitimacy claims within the political game
% \parencite{mclean02_william_h, ostrom2009understanding}. Thus, rather than
% assuming a philosopher-king stance and imposing external values as if they were
% universally agreeable, we can look at what set of values the agents themselves
% mobilize \parencite{binmore2005natural}. Particularly prominent among polarizing
% or divisive candidates -those that have strong support at the top choices of
% voters, but also have a high share of the bottom choices among the
% electorate\footnote{The related concept of an inclusive candidate is
%   defined by \textcite[p.6]{igersheim22_compar_votin_method} as those that ``get
%   widespread support from the voters but with no strong feeling of rejection or
%   attachment''.}- are claims of strength and legitimacy based on the notion of
% popular mandate, a congenial resource for politicians that, despite being
% elected, face widespread rejection or opposition.

The recognition that collective decision outcomes are intrinsically tied to the
voting procedures implemented, and that these procedures vary in their adherence
to evaluative criteria, naturally sparks curiosity about the alternative routes
that could have been pursued during pivotal electoral moments
\parencite{tabarrok1999would, kaminski1999communism, ostrom1986agenda}. At the
same time, this understanding presents a quandary: what criteria, among all the
potential ones, should guide the counterfactual analysis? Does this task not
seem inherently arbitrary, considering one could presumably retrofit a suitable
choice of voting method to correspond with a pre-determined desired outcome
\parencite{riker1982liberalism}?

A key point to anchor this discussion is the observation that political actors
themselves contemplate these procedural properties, which become instrumental in
legitimizing their position in the political sphere
\parencite{mclean02_william_h, ostrom2009understanding}. Therefore, instead of
adopting an imposing 'philosopher-king' stance and forcing external values as if
they were universally agreeable, we can observe the set of values that the
agents themselves deploy \parencite{binmore2005natural}.

This is especially prominent among polarizing or divisive candidates - those who
garner substantial support among the top choices of voters but also attract a
high proportion of the electorate's bottom choices\footnote{The associated concept of an inclusive candidate is defined by \textcite[p.6]{igersheim22_compar_votin_method} as those who "receive widespread support from the voters but with no strong feeling of rejection or attachment."}. These candidates
frequently assert claims of strength and legitimacy based on the idea of a
popular mandate \parencite{tabarrok2001president}. This notion serves as a
convenient tool for politicians who, despite being elected, confront widespread
rejection or opposition.



% Nevertheless, what is a mandate? At a minimum, a politician has a mandate as
% long as he has won under the voting procedure. An actor has more mandate the
% more significant the difference between its vote share or score vs. the second
% most well-voted candidate. Note that both notions of the mandate are related to
% a fundamental majoritarian credo which is part of the democratic ideal: that if
% both alternatives and voters are deemed equals, then the alternative that
% receives more support should be the winner \parencite{dahl1989democracy}. This
% ``monotonic/majoritarian mindset'' underlies the minimal mandate, since if a
% candidate was elected, it received more votes than the others, and the stronger
% marginal mandate, since more support means more mandate. That majoritarian value
% is what sustains claims of legitimacy of elected candidates. Roughly, the higher
% the vote margin, the more backing of popular support an Executive leader can
% claim to have \parencite{grossman2022majoritarian}. Alternatively, to deny the
% opposition has achieved a mandate by claiming the electoral process is
% fraudulent is a maneuver that again mobilizes this focal point of the democratic
% ethos.


But what exactly constitutes a mandate? At its most basic, a politician is
understood to possess a mandate as long as they have emerged victorious under
the established voting procedure. This perceived mandate is bolstered when there
is a considerable margin between the vote share or score of the winning actor
and the candidate who comes in second. It's important to note that these
interpretations of the mandate are grounded in a fundamental majoritarian
principle, a core component of the democratic ideal. This principle suggests
that if all alternatives and voters are treated equally, then the option that
garners the most support should be declared the winner
\parencite{dahl1989democracy}. This "monotonic/majoritarian mindset" is inherent
to both the ``minimal'' and the ``marginal'' mandates: the minimal mandate asserts that
if a candidate is elected, they must have received more votes than the others,
while the marginal mandate posits that greater support equates to a stronger
mandate. This majoritarian value fortifies the legitimacy claims of elected
candidates. In a general sense, the wider the vote margin, the more a political
leader can assert they have the backing of the populace
\parencite{grossman2022majoritarian}. Alternatively, discrediting an
opposition's mandate by alleging electoral fraud is a tactic that yet again
hinges on this central tenet of democratic ethos. It exploits the majoritarian
principle, suggesting that the true 'popular choice' has been undermined by an
unfair process.

However, with more than 2 alternatives, the majoritarian mindset is not as
clear-cut as a cyclical profile of voters' rankings, such as [xyz, yzx, zxy],
reminds us. Nonetheless, it remains a centerpiece of the democratic paradigm.
How can, then, one extend majoritarianism to more than two alternatives? Borda
and Condorcet grappled with this problem and gave different answers. Condorcet
extended the majority rule to pairwise majority rule: apply majority rule to all
pairwise comparisons. One possible condition that generalizes majoritarianism is
what is known as the Condorcet criterion: a decision procedure is Condorcet
consistent if it selects the candidate, if there is any, that wins in all
pairwise majority contests. This alternative is called a Condorcet winner (CW).
Borda, on the other hand, devised a scoring scheme: if there are say 3
alternatives \(\{A,B,C\}\) and an agent \(i\) has ranking \(B>C>A\) then the
Borda score in \(i\)'s ranking for each alternative is \(A:B:C = 0:2:1\).
Alternatively, it can also be coded as \(1:3:2\). The Borda score for the full
profile is the sum of each alternative score at each voter ranking, and the
candidate with the highest score, the Borda winner (BW), wins. It is equivalent
to adding the number of votes an alternative got in each pairwise comparison
against the other alternatives \parencite{nurmi1999voting}. As such, it is
another way of generalizing the ``majoritarian/monotonic'' perspective to more
than two alternatives.


Being Condorcet consistent is arguably the primary normative benchmark for a
voting method in single-candidate elections, while the Borda perspective could
be considered the leading contender \parencite{regenwetter2006behavioral,
  felsenthal2011review, nurmi2002voting}. While being plausible generalizations
of the majoritarian credo, they also offer stronger and informationally more
demanding views of mandate. If the candidate is a CW, it would have won under
all possible majority pairwise comparisons against the other candidates. I will
say a CW is a candidate with pairwise mandate. The Copeland scores could be a
more general measure of pairwise mandate, inasmuch the CW may select an empty
set, but this generalization is unnecessary in the context of this paper. The BW
lends itself to a similar interpretation, but the notion of mandate can be
strengthened here. The Borda count can be seen as one method within a family of
methods that assign weights to positions in the ballot. The higher the
proportion of positional voting systems that the candidate would have won had
the election used it, what \textcite{tabarrok2001president} has called
positional stability, the higher the positional mandate of the candidate.

Suppose a candidate wins under a voting procedure that only uses the top choice
of the electorate but is neither a BW nor a CW. In that case, it has a weaker
mandate, in this generalized majoritarian perspective, than if it were both -
which would signal a comprehensive social base. Thus, a candidate who wins under
the current voting procedure but is neither a BW nor a CW could be considered an
artifact of the procedure. In the latter case, the procedure would be just
``tracking'' a broader support pattern for the alternative. Taken together, the
concepts of pairwise and positional mandates will serve as the main analytical
tools in this paper for comprehending both the popular support a candidate
garners and the role of the decision procedure in their election.

Even though the pairwise and positional perspectives of popular support/mandate
generalize a widely held democratic principle, they are not captured by
electoral processes that only have as input voters' first choice. As such, they
are not typically mobilized by politicians. Nonetheless, this information, which
has been repeatedly rediscovered in acts of political reflexivity
\parencite{mclean14_stran_histor_social_choic_contr}, can be queried to tell a
more refined story about the backing a candidate has among the electorate. Such
a broader informational backdrop underlies current research on the case of the
United States and Donald Trump's electoral victory. Regardless of the
specificities of each paper, all presuppose that the informational paucity of
only focusing on top choices blinds the States' socio-technical translation of
popular support into political input (the choice of a candidate). For instance,
\textcite{potthoff2021condorcet, woon2020trump, kurrild2018trump} debate whether
Donald Trump was a CW in the primaries, with recommendations of voting
procedures that better track what the CW is, after all.
\textcite{igersheim22_compar_votin_method} goes a step further: they argue that
not only was Trump neither, but Sanders was the actual Borda and Condorcet
Winner, and generally the ``best'' candidate, if by best one understand to be a
candidate being the most inclusive and winning under the most alternative
decision procedures. I doubt this distribution of support for Sanders would hold
had he been a viable candidate. Moreover, changing the voting procedure implies
changing the candidates' campaign strategies. Nevertheless, this is an
intriguing result. An analogous line of reasoning would make us wonder whether a
similar conclusion could be drawn about Bolsonaro: he would not have either
pairwise or positional mandate. We will see, however, that an unambiguous
conclusion cannot be drawn in the Brazilian case.

\section{Case/Data}

Jair Messias Bolsonaro was elected as Brazil's president in 2018, following more
than 20 years as a congressman, during which he primarily served as a
low-clergy politician defending the interests of the military and
local police forces in the state of Rio de Janeiro. The 2018 electoral landscape
in Brazil was marked by widespread rejection of the traditional political elite,
especially the Labor Party (Partido dos Trabalhadores - PT). This sentiment was
fueled by recent corruption scandals and the impeachment of the previous
president, Dilma Rousseff, a member of the Labor Party.

Among the 13 contestants in the 2018 Brazilian presidential election, the main
candidates were Jair Bolsonaro, a rightist; Fernando Haddad, a leftist from the
Labor Party (PT); Geraldo Alckmin, a center-right candidate; and Ciro Gomes, a
center-left candidate. The election followed a two-round system. In the first
round, \(8.79\%\) of the votes were White/Null, meaning they were not counted,
and there was a \(20\%\) abstention rate. The result of the valid votes was as
follows: Bolsonaro received \(46.3\%\), Haddad \(29.28\%\), Ciro \(12.47\%\),
Alckmin \(4.76\%\), and Others \(7.19\%\). Among the 9 other candidates, Jo\~ao
Amo\^edo had the highest vote share at \(2.5\%\), with all others receiving less
than \(1\%\).

In the second round, the result was Bolsonaro \(55.12\%\) and Haddad
\(44.78\%\). White/Null votes constituted \(9.57\%\) of the total, and the
abstention rate was \(21.3\%\). Bolsonaro's victory, with more than a \(10\%\)
margin over his second-round opponent, was decisive. However, he contested the
result, claiming that he would have won in the first round with over \(50\%\) of
the valid votes if the elections had not been


It is important to highlight two events that significantly marked the 2018
election. First, the leading leftist candidate, Lula, was prevented from running
due to his arrest at the beginning of the electoral campaigns. This process was
later deemed suspicious in 2021, as the judge\footnote{Bolsonaro nominated this
  judge as Minister of Justice.} was found to be cooperating with the
prosecutor, and Lula won in 2022 in an electoral process marred by
irregularities favoring Bolsonaro. The support distribution for Lula was
markedly different from that for Haddad, who was merely his replacement. PT's
campaign hinged on the possibility of Lula being released, and Haddad primarily
positioned himself as Lula's candidate. However, his popularity was nowhere near
Lula's, and he inherited the high rejection of his party at the time. Second, on
September 6, 2018, a month before the first round, Bolsonaro was the target of
an assassination attempt. This knife attack likely altered his pattern of
support.


The dataset used for the analysis comes from a representative street survey
conducted on October 2, 2018, by DataFolha, an independent research institute
highly esteemed and trusted by Brazilian experts\footnote{I had access to the
  survey data, code-book, and questionnaire by creating an account and
  requesting access to them, available for educational/research purposes,
  at \url{https://www.cesop.unicamp.br}.}. This date was less than a week before
the first round of the election on October 7, 2018. The survey focused on one
question in particular, which serves as the sole variable in our analysis:
pairwise comparisons between the four top candidates. These comparisons
included: Alckmin \(\times\) Bolsonaro, Alckmin \(\times\) Ciro, Alckmin
\(\times\) Haddad, Bolsonaro \(\times\) Ciro, Bolsonaro \(\times\) Haddad, and
Haddad \(\times\) Ciro, with no option for indifference.

With these strict comparisons for all survey respondents, it is possible to
reconstruct their full ranking of the top four candidates. Preliminary
pre-processing led me to drop 171 observations where all pairwise comparisons
were missing and 132 where they were cyclic, leaving 2937 out of 3240
observations. As Table~\ref{Tab:Tcpairwise} shows, only 1797 observations
compared all four candidates. Therefore, we had to augment the data with
transitive closures for 1140 observations, using methods discussed in the next
section.




% \begin{table}[]\centering \resizebox{0.5\columnwidth}{!}{
% \begin{tabular}{|l|r|} \hline Number of Pairwise Comparisons & Frequency \\
% \hline 1 & 15 \\ \hline 2 & 42 \\ \hline 3 & 462 \\ \hline 4 & 118 \\ \hline 5 &
% 503 \\ \hline 6 & 1797 \\ \hline
% \end{tabular} }
% \caption{Frequency of pairwise comparisons in the dataset.}

% \label{Tab:Tcpairwise}
% \end{table}

\begin{table}[!h]\centering
\begin{tabular}{lr}
\hline
Number of Pairwise Comparisons & Frequency \\ \hline
1                              & 15        \\
2                              & 42        \\
3                              & 462       \\
4                              & 118       \\
5                              & 503       \\
6                              & 1797      \\ \hline
\end{tabular}


\caption{Frequency of pairwise comparisons in the dataset.}

\label{Tab:Tcpairwise}
\end{table}


\section{Methods}

I impute the missing data using the \textbf{\textsf{R}} package
\(\operatorname{mice}\) (multiple imputation by chained equations), one of the
standard packages for this task. It fills the missing values in a row by using
the values of the other columns, by an iterative series of predictive models
\parencite{vanbuuren2018imputation}. Under the hood, it offers a menu of
possible predictive models, such as bayesian linear regression, predictive mean
matching, logistic regression, polytomous regression, classification trees and
random forests. Among the classes of methods that could be applied to the
missing voting data, given its categorical nature, the polytomous regression was
the only one that did not introduce cyclic rankings, or repeated alternatives in
the ranking, and as such, was the one I used\footnote{Besides the polytomous
regression, I tested predictive mean matching, classification trees, and random
forests. All introduced cyclic rankings, sometimes in large amounts (as in the
case of random forests).}.


A further complication is a mismatch between the survey's plurality result and
the actual result of the first round. This is typical in surveys and might be
due to strategic voting, social desirability bias (not wanting to be seen as
``extreme''), or systematic refusal of part of the electorate to answer the
survey \parencite{nishimura2016alternative}. Any imputation technique will
reproduce this top-choice discrepancy since it inherits this problem from the
survey. The share in the survey is Bolsonaro:Haddad:Ciro:Alckmin:Others =
\(36.81 : 24.96 : 17.06: 13.97 : 7.2 \). Thus, Bolsonaro and Haddad are
undervoted in the sample, while Ciro and Alckmin are overvoted\footnote{Remember the actual result was Bolsonaro:Haddad:Ciro:Alckmin:Others = \(46.3 : 29.28 : 12.47 : 4.76 : 7.19 \).}. However,
transferring is complicated by the fact that we are working with the full
rankings, which gives leeway to many possible ways of transferring. To solve
this problem I develop a systematic transfer method, as discussed in the next
subsection.

\subsection{The Transfer Method}
To get a hold of the problem,  consider Table
~\ref{tbl:overunderex}, which shows some of the rankings for Alckmin and
Bolsonaro after the imputation. If we are to transfer from Alckmin to Bolsonaro,
we are led to the problem of first picking which ranking at the source should be
chosen and then which ranking at the target should receive votes while
respecting how much the source has in excess and how much the target needs.
Which row from the set \(\{1,2,3\}\) should transfer votes to which row of the
set \(\{4,5,6\}\)?

\begin{table}[!h] \centering %\resizebox{0.8\columnwidth}{!}{
\begin{tabular}{rllllrr} \hline & 1 & 2 & 3 & 4 & frequency & proportion \\
\hline 1 & Alckmin & Bolsonaro & Ciro & Haddad & 93 & 0.03 \\ 2 & Alckmin & Ciro
& Bolsonaro & Haddad & 63 & 0.02 \\ 3 & Alckmin & Haddad & Bolsonaro & Ciro & 14
& 0.00 \\ 4 & Bolsonaro & Alckmin & Ciro & Haddad & 556 & 0.18 \\ 5 & Bolsonaro
& Ciro & Alckmin & Haddad & 366 & 0.12 \\ 6 & Bolsonaro & Alckmin & Haddad &
Ciro & 59 & 0.02 \\ \hline
\end{tabular} %}
\caption{Some pre-transfer proportions of Alckmin/Bolsonaro's rankings}
\label{tbl:overunderex}
\end{table}

% A natural sorting of which ranking should be the source is the position
% Bolsonaro is in the ranking. We start with rankings in which he is in the second
% position ((Alckmin, Bolsonaro, Ciro, Haddad), (Alckmin, Bolsonaro, Haddad,
% Ciro)), then third position ((Alckmin, Ciro, Bolsonaro, Haddad), (Alckmin,
% Haddad, Bolsonaro, Ciro )), then last position ((Alckmin, Ciro, Haddad,
% Bolsonaro), ((Alckmin, Haddad, Ciro, Bolsonaro))).

% Suppose we picked a source ranking from the first sorted rankings set. What
% should be the target ranking among the rankings which have Bolsonaro as the
% first choice? I transfer to the ranking that has minimal Kemeny's distance to
% the source ranking \parencite{nurmi2002voting}. The Kemeny distance counts the
% number of transpositions (switching of pairs) needed to go from one permutation
% to another permutation. Thus, I transfer from the source ranking the
% \(\operatorname{\mathbf{min}}\)(number of votes the source ranking has, the
% total number of votes the under-voted needs, the total number of votes the
% over-voted can give)\footnote{The \(\operatorname{\mathbf{min}}\) guarantees: we
% are not giving more than the source ranking has, which would lead to negative
% numbers; less or more than the undervoted needs; nor giving more than the
% over-voted should overall give (at some iteration in the loop, a ranking can
% have a higher frequency than both what the over-voted can give and the
% under-vote needs to receive).}. I update the source ranking frequency, the
% target ranking frequency, the total number of votes the under-voted needs, and
% the total number of votes the over-voted can give. If the under-voted does not
% need any other votes, the algorithm breaks the loop and goes to another
% over-voted \(\to\) under-voted transfer. If not, it checks if the over-voted can
% still transfer votes to the current target under-voted. If yes, it picks another
% source ranking in the sorted rankings sets and repeats until the source has run
% out of votes it can give or the target has received enough votes. If not, it
% goes to another over-voted \(\to\) under-voted transfer. In the end, this leads
% to 24 possible transfer sequences from over-voted to under-voted. One possible
% sequence is Alckmin \(\to\) Bolsonaro, then Alckmin \(\to\) Haddad, then Ciro
% \(\to\) Haddad, then Ciro \(\to\) Bolsonaro. Another possible sequence is
% Alckmin \(\to\) Bolsonaro, then Alckmin \(\to\) Haddad, then Ciro \(\to\)
% Bolsonaro, then Ciro \(\to\) Haddad. That transference process leads to 6
% transfers that minimize the Euclidean distance between the inferred plurality
% result and the actual result of the first round. The results are invariant
% between them, so I only report the analysis for one of them. The new percentages
% are: Bolsonaro:Haddad:Ciro:Alckmin:Others = \(46.32:29.26:12.45:4.77:7.19 \).

% After imputing the missing rankings and making the transfer of rankings to match
% the result for the first round, I identify the BW and CW among the top 4
% candidates, calculate and plot all counterfactual victory scenarios for
% positional voting methods, and visualize the positional outcomes for alternative
% 3-candidates sets using Saari's outcome simplex, which I discuss in the next subsection.


The process of sorting and transferring rankings can be understood through a systematic approach. We begin by organizing the rankings based on Bolsonaro's position. The rankings are sorted into three sets: those where he is in the second position (e.g., (Alckmin, Bolsonaro, Ciro, Haddad), (Alckmin, Bolsonaro, Haddad, Ciro)), third position (e.g., (Alckmin, Ciro, Bolsonaro, Haddad), (Alckmin, Haddad, Bolsonaro, Ciro)), and last position (e.g., (Alckmin, Ciro, Haddad, Bolsonaro), (Alckmin, Haddad, Ciro, Bolsonaro)).

Suppose we select a source ranking from the first sorted set (second position).
We then need to determine the target ranking among those where Bolsonaro is the
first choice. The target is chosen based on the minimal Kemeny's distance to the
source ranking \parencite{nurmi2002voting}. The Kemeny distance measures the
number of transpositions (switching of pairs) needed to transform one
permutation into another. The transfer from the source ranking is then the
\(\operatorname{\mathbf{min}}\)(number of votes the source has, total votes the
under-voted needs, total votes the over-voted can give)\footnote{This
  \(\operatorname{\mathbf{min}}\) ensures that the transfer does not lead to
  negative numbers, nor gives more or less than needed.}. The frequencies of the
source and target rankings are then updated, along with the total votes needed
by the under-voted and the total votes the over-voted can give. I update the
source ranking frequency, the target ranking frequency, the total number of
votes the under-voted needs, and the total number of votes the over-voted can
give. If the under-voted does not need any other votes, the algorithm breaks the
loop and goes to another over-voted \(\to\) under-voted transfer. If not, it
checks if the over-voted can still transfer votes to the current target
under-voted. If yes, it picks another source ranking in the sorted rankings sets
and repeats until the source has run out of votes it can give or the target has
received enough votes. If not, it goes to another over-voted \(\to\) under-voted
transfer.


In the end, this leads to 24 possible transfer sequences from over-voted to
under-voted. For example, one sequence might be Alckmin \(\to\) Bolsonaro, then
Alckmin \(\to\) Haddad, then Ciro \(\to\) Haddad, then Ciro \(\to\) Bolsonaro.
Another sequence might differ in the order or targets. The entire transference
process results in six transfers that minimize the Euclidean distance between
the inferred plurality result and the actual result of the first round. Since
the results are invariant between them, only one analysis is reported. The new
percentages are: Bolsonaro:Haddad:Ciro:Alckmin:Others =
\(46.32:29.26:12.45:4.77:7.19\).

Having imputed the missing rankings and transferred them to match the results of
the first round, I identify the Borda Winner (BW) and Condorcet Winner (CW)
among the top four candidates. Then, I calculate and plot all counterfactual
victory scenarios for positional voting methods. Finally, I visualize the
positional outcomes for alternative sets of three candidates using Saari's
outcome simplex, a concept that I will elaborate on in the next subsection.

\subsection{Saari's Outcome Simplex}

Positional voting methods are a class of voting systems that assign scores to
positions in voters' preferences. These methods are characterized by the
principle that the higher an alternative is ranked in a voter's preference, the
higher the score it receives. A better-ranked alternative should thus receive a
score at least as large as the next best-ranked alternative.

Donald Saari has proved several facts that help in understanding and calculating
positional voting victories. Any positional voting method that respects the
above constraint can be defined by a vector of weights
\(W_{j} = (w_{1}, \ldots, w_{n})\). For example, in a three-candidate election,
plurality voting can be represented by the vector \((1,0,0)\), while the Borda
count is represented by \((3,2,1)\). It's important to note that plurality could
also be coded as \((2,0,0)\) or even \((2,1,1)\), and the result would not
change. Similarly, the Borda count could be associated with the vector
\((2,1,0)\), with no change in the outcome.

This flexibility in representation comes from an equivalence relation: any two
weighting vectors \(W_{1}\) and \(W_{2}\) that satisfy
\(W_{1} = aW_{2} + b(1,\ldots, 1)\) for \(a,b > 0\) will have the same result in
terms of outcome ordering \parencite{saari1995basic}. This relation can be
exploited to standardize positional voting methods.

For four candidates, the weighting vectors can be expressed as
\((1,s_{1},s_{2},0)\), where \(0 \leq s_{2} \leq s_{1} \leq 1\). In plurality,
\(s_{1} = s_{2} = 0\), while in the Borda count, \(s_{1} = \frac{2}{3}\) and
\(s_{2} = \frac{1}{3}\). For three candidates, all such positional voting
methods will lie on the line connecting plurality with antiplurality
(\(1,1,0\)). For four candidates, all such procedures will lie in the convex
hull of plurality, antiplurality, and vote-for-two procedures, with respective
weights of \((1,0,0,0), (1,1,1,0), (1,1,0,0)\). Calculating scenarios, thus,
amounts to varying the values of \(s_{1}\) and \(s_{2}\)
\parencite{saari1995basic, saari2001chaotic}.


\begin{figure}[!h] \centering
  \begin{subfigure}[b]{0.49\textwidth} \centering
\includegraphics[width=\textwidth]{./images/simpletriangle.png}
 \caption{Point-share triangle}
 \label{fig:pointshare}
\end{subfigure} \hfill
  \begin{subfigure}[b]{0.49\textwidth} \centering
\includegraphics[width=\textwidth]{./images/representation_triangle_noth}
 \caption{Profile triangle}
 \label{fig:representation}
\end{subfigure}
\caption{Saari's triangles}
\label{fig:saari_nurmi}
\end{figure}

% \begin{figure}[H] \centering \includegraphics[width=0.8\columnwidth,
% height=0.3\textheight]{./images/simpletriangle.png}
%  \caption{Saari's outcome simplex}
%  \label{fig:saari_nurmi}
% \end{figure}

Saari's outcome simplex, or point-share triangle
\parencite{eggers20_diagr_analy_ordin_votin_system}, makes use of the structure
of positional voting methods to provide a way of visualizing all possible
positional voting results of an election\footnote{For a complete exposition of
  this method see \textcite{saari1995basic} or \textcite{nurmi2002voting}.}.
Consider Figure \ref{fig:pointshare}. The closer to a vertex, the better the
vertex's position in the social ranking. Region 1 corresponds to the social
ranking of \(A > B > C\), while Region 4 corresponds to the social ranking of
\(C>B>A\). The lines separating the regions represent indifference. The point at
which all lines meet corresponds to \(A \sim B \sim C\), while the line
separating Region 1 and 2 would correspond to \(A > B \sim C\). The three dots
are the results of the antiplurality, the Borda and the plurality voting
methods. The Borda Count point is always closer to the antiplurality result. In
this example, most positional voting methods would have agreed with the
plurality procedure outcome of \(B\) as the winner. A related triangle, the
representation triangle, or profile triangle
\parencite{eggers20_diagr_analy_ordin_votin_system}, will be used to represent a
profile compactly. Figure \ref{fig:representation} shows the equivalent profile
triangle. In each ranking region, we plot the frequency of votes that match that
region's ranking. For 4 candidates, we can use analog representations by
``opening'' the 3-simplex/tetrahedron and plotting onto its polyhedral net - the
arrangement of polyhedrons in the plane that, when folded, become the faces of
the simplex.

\section{Results} The inferred rankings are shown in Figure \ref{fig:rep_ot} and
summarized in Figure \ref{fig:counts}. Among all ways of transferring from
over-voted to under-voted, while respecting the Kemeny distance, the
transference that best matched the rankings in the survey with the actual
first-round result led all rankings in which Alckmin appears as the first choice
to be of type Alckmin > Ciro > Haddad > Bolsonaro and Alckmin > Haddad > Ciro >
Bolsonaro. Notably, no ranking of type Alckmin > Bolsonaro > \(\textunderscore\)
> \(\textunderscore\) remained. The most blatant pattern in Figure
\ref{fig:counts} is that the candidates that went to the second round were,
indeed, the most divisive ones. Among the more inclusive candidates, Ciro has
more second choices than third choices, while Alckmin's support is equally
balanced between those two positions in the voters' preferences. Moreover, Ciro
has more first choices and fewer last choices than Alckmin. There is also a
difference among the divisive candidates: Haddad's rejection was higher than his
top-choice support, while the opposite held for Bolsonaro. That within-group,
inclusive vs. divisive, differences will be relevant to understand how each
candidate fares against the others.

\begin{figure}[!h] \centering
  \begin{subfigure}[b]{0.8\textwidth} \centering
\includegraphics[width=\textwidth]{./images/representation_tetrahedron.png}
 \caption{Opened representation tetrahedron}
 \label{fig:rep_ot}
\end{subfigure} \hfill
  \begin{subfigure}[b]{0.8\textwidth} \centering
\includegraphics[width=\textwidth]{./images/corrected1_indexes_plot.png}
 \caption{Frequencies at each position in the rankings}
 \label{fig:counts}
\end{subfigure}
\caption{Profile after imputation and rankings transference}
\label{fig:profile_trans}
\end{figure}

What does this support distribution mean from the point of view of the BW and
CW? Table \ref{tbl:overallresult} shows what we can infer from the imputed data.
Recall we're using data from a survey that happened before the first round, and
will extrapolate to consider what could have happened in general. Despite being
divisive, Bolsonaro would have won in all pairwise majority comparisons against
other top candidates. Haddad, however, would have lost against all others. He
was a Condorcet Loser, despite going to the second round. Ciro would only have
lost against Bolsonaro, while Alckmin could only have won against Haddad. Unlike
what was widely believed at the time and was the motto of his campaign, it is
uncertain whether Ciro would have won against Bolsonaro in the second round.
From a pairwise perspective, he was not the ``anti-Bolsonaro'', but merely an
``anti-Haddad'', even more than Bolsonaro. Alckmin, the candidate with the
longest television time and the broadest supporting coalition, would have lost
against Haddad, who was merely a substitute for Lula. However, the pattern is
not reflected in the Borda Scores, which implies the ranking: Ciro > Bolsonaro >
Alckmin > Haddad. Nevertheless, the raw Borda scores of Ciro and Bolsonaro are
very similar. If we standardize them, we see that the candidates are practically
tied. If we take into account the sampling error, imputation, and transfer
degree of freedom, then the most we should conclude is that the Borda Ranking
was Ciro \(\sim\) Bolsonaro > Alckmin > Haddad. Note that if we take a
positional perspective, then yes, Ciro was indeed the main contestant against
Bolsonaro. Nevertheless, this obviously could not be captured by the majority with run-off, which excluded him from the second round.

\begin{table}[!h]
  \begin{adjustwidth}{3cm}{}
\begin{subtable}[t]{0.48\textwidth} \centering
          \begin{tabular}{rrrrr} & Alckmin & Bolsonaro & Ciro & Haddad \\\hline
Alckmin & 0.0\% & -12.63\% & -16.99\% & 8.27\% \\ Bolsonaro & 12.63\% & 0.0\% &
5.48\% & 7.46\% \\ Ciro & 16.99\% & -5.48\% & 0.0\% & 16.65\% \\ Haddad &
-8.27\% & -7.46\% & -16.65\% & 0.0\% \\\hline
          \end{tabular}
    \caption{Pairwise Margins}
     \label{tbl:margins}
   \end{subtable} \vspace*{0.3cm}

\begin{subtable}[t]{0.48\textwidth} \centering
\begin{tabular}{rrr} \hline & Borda Score & Standardized Borda Score\\ \hline
Alckmin & 7029 & 0.464 \\ Bolsonaro & 7718 & 0.543 \\ Ciro & 7756 & 0.547\\
Haddad & 6867 & 0.446 \\ \hline
\end{tabular}
\caption{Borda Count Outcome}
\label{tbl:Borda}
\end{subtable}
\end{adjustwidth}
\caption{Condorcet and Borda Outcomes}
\label{tbl:overallresult}
\end{table}

Now, what about the positional mandate? As discussed in the methods section,
with 4 candidates, all results will lie in the convex hull of three positional
voting procedures: plurality, antiplurality, and vote for two. The normalized
score of a candidate will be of the form
\(q_{s_{i}} = a_{i} + b_{i}s_{1} + c_{i}s_{2}\), where \(a_{i}\) is the share
\(i\) received of votes in the first position, \(b_{i}\) in the second, and
\(c_{i}\) in the third position of voters rankings. Therefore, the scores of
each candidate in the inferred ranking for the 2018 election can be found by
assigning values to the equations of Table ~\ref{tbl:ws}. For instance, if we
set \(s_{1} = s_{2} = 0\) we recover the plurality score, after ignoring
``Other'' candidates.

\begin{table}
  \begin{tabular}{rr}
    \hline\hline
    \textbf{candidates} & \textbf{w_s tallies} \\
    \texttt{String} & \texttt{SymPy.Sym} \\\hline
    Alckmin & 0.4075*s₁ + 0.4261*s₂ + 0.0517 \\
    Bolsonaro & 0.0392*s₁ + 0.0459*s₂ + 0.4985 \\
    Ciro & 0.4402*s₁ + 0.3524*s₂ + 0.1345 \\
    Haddad & 0.1131*s₁ + 0.1756*s₂ + 0.3153 \\\hline\hline
  \end{tabular}
\end{table}



We can, then, represent the results by an opened outcome tetrahedron, as roughly
depicted in Figure~\ref{fig:ot}. As depicted in the legend, the black upside
triangle in the triangle close to the B vertex is the plurality result, the
black downside triangle in the triangle close to the C vertex is antiplurality,
the black dot that is close both to the C and B vertex is the vote for two
results, and the diamond that touches the line segment from A to H is the Borda
count. To easen interpretation, Figure \ref{fig:ot_win} shows the areas in which
each candidate would have been top ranked, while Figure \ref{fig:ot_second}
shows the areas in which each candidate would appear in the second position in
the social ranking. As expected, the decision procedures that emphasize the top
choice awarded Bolsonaro and Haddad to the extent that the Condorcet loser went
to the second round. Note, however, that Haddad would be second placed only in a
small region of the hull. Moreover, we can see that there are decision
procedures in which even Alckmin would have beaten both Bolsonaro and Haddad,
precisely those procedures in which Ciro would be top ranked while Alckmin would
have been second ranked, in the social ranking. Figure \ref{fig:ot_win} also
shows that the procedure hull is completely contained in areas where Ciro and
Bolsonaro would be top ranked. If we look simultaneously at both Figure
\ref{fig:ot_win} and Figure \ref{fig:ot_second} we can have a more thorough
understanding of the positional scenario in the 2018 election. There is a small
region in which Bolsonaro is top ranked and Haddad is second ranked, than a
large area in which Bolsonaro is still top ranked and Ciro is second ranked,
then in some procedures Ciro is top ranked and Bolsonaro is second ranked, and
finally a small area where Ciro would be top ranked while Alckmin would be
placed second in the social ranking.

% \begin{figure}[!h] \centering \includegraphics[width=0.5\columnwidth,
% height=0.3\textheight]{./images/opened_tetrahedron1.png}
%  \caption{Saari's opened tetrahedron }
%  \label{fig:ot}
% \end{figure}

\begin{figure}[!H] \centering
  \begin{subfigure}[b]{\textwidth} \centering
\includegraphics[width=\textwidth]{./images/top_ranked.png}
 \caption{Winning regions}
 \label{fig:ot_win}
\end{subfigure} \hfill
  \begin{subfigure}[b]{\textwidth} \centering
\includegraphics[width=\textwidth]{./images/second_ranked.png}
 \caption{Regions candidate is second ranked}
 \label{fig:ot_second}
\end{subfigure}
\caption{Saari's opened tetrahedron}
\label{fig:ot}
\end{figure}

In what precise percentage of the cases would a candidate have beaten another?
Note that by opening the tetrahedron, the information provided by the volume of
the subregions of the procedure hull is lost. Nevertheless, algebraic
manipulation of the equations in Table \ref{tbl:ws} allows us to answer this
question. For instance, when is \( 0.4113 s_{1} + 0.4165 s_2 + 0.0514 >
0.0392 s_{1} + 0.0521 s_2 + 0.4992 \) given the aforementioned constraints
on the values of the parameters? Solving this inequality gives us the intervals
of the values of the parameters in which Alckmin would have beaten Bolsonaro in
the set of positional voting procedures. The area of those intervals over the
area of all the combinations gives us the percentage of scenarios in which a
candidate could beat another, as shown in Table ~\ref{tbl:ctn} while the actual
figure whose area is being used is shown in Figure~\ref{fig:positional4c}.

Table ~\ref{tbl:ctn} implies a more complex picture of what happened. Bolsonaro
was the CW, was tied with Ciro as a BW and would have won against Ciro in
roughly 47\(\%\) of the positional voting methods. Nevertheless, it shows that
there were scenarios in which he would have lost to the more inclusive
candidates, Ciro and Alckmin. In Alckmin's case, this could have happened in
surprising \(30\%\) of the cases. However, Ciro could have beaten him in most,
\(\approx 53\%\), of the positional voting methods. Surprisingly, Haddad, who
went to the second round with Bolsonaro, would never have beaten him. The
explanation for that is the following: as shown in Figure \ref{fig:counts},
Haddad and Bolsonaro were both divisive candidates, but Bolsonaro had more
support than Haddad. They were not equally supported/rejected. Given that they
were both divisive, most of their support was in the top choice, they would have
fared equally well or badly under the same positional voting methods, but since
Bolsonaro had more first votes and was less frequently in the bottom of the
rankings than Haddad he actually ``positionally dominated'' Haddad. The same
logic applies to another surprising result: Alckmin would never have beaten
Ciro.

%% \begin{table}
  \begin{tabular}{rrrrr}
    \hline\hline
    \textbf{candidates} & \textbf{Alckmin} & \textbf{Bolsonaro} & \textbf{Ciro} & \textbf{Haddad} \\
    \texttt{String} & \texttt{Float64} & \texttt{Float64} & \texttt{Float64} & \texttt{Float64} \\\hline
    Alckmin & 0.0 & 0.32 & 0.0 & 0.57 \\
    Bolsonaro & 0.68 & 0.0 & 0.47 & 0.98 \\
    Ciro & 1.0 & 0.53 & 0.0 & 0.8 \\
    Haddad & 0.43 & 0.02 & 0.2 & 0.0 \\\hline\hline
  \end{tabular}
\end{table}


\begin{table}[H]
  \centering
  \begin{tabular}{rrrrr}
    \hline
     & Alckmin & Bolsonaro & Ciro & Haddad \\
    \hline
    Alckmin & 0.0 & 0.31 & 0.0 & 0.58 \\
    Bolsonaro & 0.69 & 0.0 & 0.47 & 1.0 \\
    Ciro & 1.0 & 0.53 & 0.0 & 0.81 \\
    Haddad & 0.42 & 0.0 & 0.19 & 0.0 \\\hline\hline
  \end{tabular}
   \caption{Proportion of victories in the positional voting procedure set}
\label{tbl:ctn}
\end{table}

Naturally, proportions do not show what were the decision procedures in which,
for instance, Ciro would have beaten Bolsonaro. Intuitively, voting procedures
that emphasize rejection or more of the middle region of the rankings should
give an advantage to inclusive candidates, which is qualitatively confirmed by
Figure~\ref{fig:ot}. Since the positional voting methods with four candidates
are determined by their \(s_{1}\) and \(s_{2}\) weights, we can visualize all
scenarios by varying them, as in Figure~\ref{fig:positional4c}. It shows the
scenarios Bolsonaro \(\times\) Ciro, Ciro \(\times \) Haddad, and Alckmin
\(\times\) Bolsonaro. Note that, as expected, the only way Alckmin could have
beaten Bolsonaro would be if \(s_{1}\) and \(s_{2}\) were above 0.6. Recall that when both are 1, the voting procedure is antiplurality, a method equivalent
to saying which candidate voters dislike. However, this universe of cases was dominated by Ciro, who would have beaten Bolsonaro in any combination of
\(s_{1}\) and \(s_{2}\) higher than the line connecting the points
\((0.51,0.51)\) and \((0.9,0)\). The plot also shows what combinations of
weights lead to \(81\%\) of Ciro \(>\) Haddad: any combination to the right of
the line segment connecting \((0.35,0.35)\) and \((0.55,0.0)\).

\begin{figure}[!h] \centering \includegraphics[width=\columnwidth,
height=0.3\textheight]{./images/counterfactual_triangle.jpg}
\caption{Pairwise victory for three selected pairs of candidates}
 \label{fig:positional4c}
\end{figure}


Nonetheless, the family of positional voting methods does not satisfy, in
general, Independence of the Alternative Set \parencite{kaminski2015empirical}.
If we drop or add candidates, the ``social'' ranking might change without
respecting the ordering of the baseline set of alternatives. Consider the
Borda-induced social ranking in this case: Ciro \( \sim \) Bolsonaro > Haddad >
Alckmin. If by dropping Alckmin, the ranking changes to Bolsonaro > Haddad >
Ciro > Alckmin, then the Borda Count, in this case, would be inducing a
``paradoxical'' result. In Figure ~\ref{fig:c1dropping}, I consider alternative
scenarios by dropping one of the top 4 candidates.

The positional voting procedures are eminently well-behaved when dropping
candidates from this dataset. There is a minor tilt toward Bolsonaro winning
with the Borda Count in Figure~\ref{fig:notah1}, but as I have previously
argued, this seems like a tie, given the underlying uncertainty. Notice that in
all scenarios where Bolsonaro is still in the alternative set, he would have
been the plurality winner. However, he would have tied with Ciro under Borda and
lost against him with decision procedures that put more weight on rejection, as
in the 4 candidates analysis. In the scenario Ciro was not in the set, Bolsonaro
would again only have lost against Alckmin, but in a minority of cases.


\begin{figure}[!h] \centering
        \begin{subfigure}[b]{0.475\textwidth} \centering
\includegraphics[width=\textwidth]{./images/cw1_nota.png}
             \caption{}% % {{\small Network 1}}
            \label{fig:notac1}
        \end{subfigure} \hfill
        \begin{subfigure}[b]{0.475\textwidth} \centering
\includegraphics[width=\textwidth]{./images/cw1_notb.png}
             \caption{}% % {{\small Network 2}}
            \label{fig:notbc1}
        \end{subfigure} \vskip\baselineskip
        \begin{subfigure}[b]{0.475\textwidth} \centering
\includegraphics[width=\textwidth]{./images/cw1_notc.png}
            \caption{}%
            \label{fig:notcc1}
        \end{subfigure} \hfill
        \begin{subfigure}[b]{0.475\textwidth} \centering
\includegraphics[width=\textwidth]{./images/cw1_noth.png}
             \caption{}% % {{\small Network 4}}
            \label{fig:notah1}
        \end{subfigure}
        \caption[ Positional results when dropping one candidate ] {\small
Positional results after dropping one candidate }
        \label{fig:c1dropping}
    \end{figure}

    We have seen that besides having a high vote margin, Bolsonaro was also a
    CW. His victory, thus, was not a fluke or an artifact of institutional
    technology. However, he was probably not the candidate with the strongest
    positional mandate. This result revisits the Borda \(\times\) Condorcet
    controversy. On the one hand, he was the CW, the primary normative benchmark
    for a voting procedure. On the other hand, in the Brazilian case, the Borda
    count would have been a more substantial barrier against a divisive
    candidate. Even though we could expect divisive candidates to have fared
    worse under informationally richer decision procedures, a divisive candidate
    can still be a CW with \(47\%\) positional stability/mandate. Yes, Ciro
    would have won against him in \(53\%\) of the positional methods, and at
    least would have tied with him in the Borda count, but most of the methods
    within this \(53\%\) emphasize rejection, and to give more weight to
    rejection vis-\`a-vis approval seems unreasonable under any set of normative
    expectations demanded of a decision procedure for large-scale democratic
    elections. Due to its symmetry, the Borda Count lies at a threshold: its
    constant decrease in assigning weights to the positions in the rankings
    guarantees that approval matters more than rejection, but without throwing
    away the rejection information. Therefore, highly polarized scenarios can
    lead to the election of a divisive candidate, which puts in dispute two
    reasonable metrics of support: being a CW vs. being a BW. This means the
    paper only gives partial support to the hypothesis that informationally
    richer decision procedures would be enough to contain divisive candidates,
    and two reasonable generalizations of the majoritarian credo end up in
    conflict.

    However, Figure~\ref{fig:notbc1} presents an interesting scenario. Here,
    there is no conflict between the perspectives: under both positional and
    pairwise perspectives on a mandate, Haddad going to the second round was
    purely an institutional fluke. Even though it looks like Haddad would be a
    plurality winner if we dropped Bolsonaro, the plurality point is close to a
    tie with Ciro, and could be read as such, given the uncertainty. Ciro, thus,
    now would have almost 100\(\%\) positional mandate. Moreover, in Table
    \ref{tbl:margins} it was shown that Ciro would have beaten him with a
    majority pairwise comparison, which gives credence to affirming that Ciro
    would have won under a majority with run-off system. In this scenario, the
    most inclusive candidate would have been elected. Ceteris paribus, it seems
    the only way an agreement between the Borda and Condorcet criteria could be
    guaranteed to exist in the Brazilian 2018 case would be if Bolsonaro had
    never been able to run. In this scenario, the Condorcet Loser would again be
    beaten, but now a candidate with a solid mandate, as endorsed both by the
    pairwise majority comparisons and the entire hull of positional methods
    would have been elected.

\section{Conclusion}

In this paper, I sought to determine whether Bolsonaro's victory in 2018 was
merely a fluke or an accident resulting from the historically contingent
decision procedure in use at the time. To answer this question, we first sought
to understand what would constitute more than just an accidental outcome. I
argued that the notions of pairwise and positional mandate can be derived from
well-established axiological perspectives to evaluate whether an electoral
result is solely an institutional fortuity. Then, I demonstrated that even
though the aggregation procedure boosted Bolsonaro's victory, it was not merely
its effect, contrary to established theoretical expectations, but neither was he
an undisputed winner under both aforementioned evaluative criteria for gauging
the mandate of a candidate in democratic collective choice situations.

In terms of future research, contrasting Haddad with Lula and analyzing the
effect of the knife attack should provide a more comprehensive picture of what
happened in 2018. Moreover, the pipeline for the analysis is highly
reproducible. Analogous analysis can be done for any case with majority with
run-off presidential elections, or even to any survey that contain pairwise
comparisons between the top candidates, as many in Brazil do. As such, any other
majoritarian election in Brazil could be analogously analyzed. Figure~\ref{fig:notbc1} could
also be the starting point for an investigation of the selection of the pool of
candidates allowed to compete for the Executive, particularly in transitional
democracies, given Brazil's lax transitional justice, and Bolsonaro's intimate
connection with the remnants of the Dictatorship. First, from a positive point
of view as a causal pathway for democratic backsliding
\parencite{svolik2008authoritarian, nalepa2022after}. Second, from a normative
perspective. What, if anything, justifies restricting classes of actors from
running for certain positions? What values would conflict here?

A certain limitation of that paper is that I used only one variable from
the dataset, pairwise comparisons, to simulate alternative scenarios. However,
socio-demographic variables from the dataset could have been used to strengthen
the data imputation procedure. Moreover, roughly less than half of the dataset
is constituted of incomplete pairwise comparisons, and there may be valuable
information on the agent's preferences contained in patterns of missingness.

Another limitation is that agents adapt to new institutional environments. I am
ignoring strategic voting by assuming a direct translation between preferences
and behavior. However, the percentage of strategic voting in a large-scale
election is an open empirical problem
\parencite{straeten10_strat_sincer_heuris_votin_under,kawai2013inferring}.
Nevertheless, a combination of game-theoretic models with a simulation
parameterized by the inferred ranking distribution is a route of research that
could be pursued.

Though I have analyzed the four top candidates, there can be discrepancies when
we have a subset of the alternatives vs. when we have the whole set of
candidates \parencite{saari2001chaotic, kaminski2015empirical}. It is
well-known, for instance, that the Borda Count is susceptible to the
winner-turns-loser paradox. Finally, even though I have analyzed scenarios in
which candidates were removed, it would have been more realistic to simulate the
formation of coalitions and how voters would have reacted to that
\parencite{kaminski1998revival}. The assumption of a pure additive transfer of
votes, implicit when we removed candidates, is not necessarily valid with
coalitions \parencite{kaminski2001coalitional}, insofar voters of a center-left
candidate, for instance, could vote for the center-right candidate if they are
alienated by an alliance with the Left, which, in the case of the election under
scrutiny, was highly rejected. \printbibliography





\end{document}
