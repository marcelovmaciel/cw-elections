\documentclass{beamer}
\usetheme{metropolis}           % Use metropolis theme
\usepackage[utf8]{inputenc}
%\usepackage[brazil]{babel}
\usepackage[normalem]{ulem}
\usepackage{float}
\usepackage{caption}
\usepackage{subcaption}
\usepackage{graphicx}
\usepackage{xcolor}
\usepackage[backend = biber]{biblatex}
%\addbibresource{refs.bib}
\metroset{background=light}
\addbibresource{refs.bib}
\usepackage{tikz}
\def\checkmark{\tikz\fill[scale=0.4](0,.35) -- (.25,0) -- (1,.7) -- (.25,.15) -- cycle;}
\title{Majoritarian principles in critical junctures: an analysis of Brazil's
  2018 presidential election}

\date{}
\author{Marcelo Veloso Maciel}
\institute{University of California, Irvine}
\begin{document}
\maketitle

% ----------------- NOVO SLIDE --------------------------------
\begin{frame}
\frametitle{Context : electoral success of highly divisive candidates}
  \begin{figure}[H] \centering \includegraphics[width=\textwidth]{./trumpolnaro.png}
 %\caption{Jair Messias Bolsonaro and Donald Trump}
 \end{figure}
\end{frame}

\begin{frame}
  \frametitle{Research Question }
  Is the election of divisive or polarizing candidates an artifact of the voting
  methods?
\end{frame}

\begin{frame}
  \frametitle{Prior research}
  \begin{itemize}
    \item \textcite{potthoff2021condorcet} and \textcite{kurrild2018trump} argue
          that Trump might have been a Condorcet loser. \textcite{woon2020trump}
          argue he was in the Core.
    \item \textcite{igersheim22_compar_votin_method} argue that the
          Condorcet,Borda, Utilitarian winner was actually Sanders.
  \end{itemize}
\end{frame}

\begin{frame}
  \frametitle{Hypothesis}
  I expected similar results in the Brazilian 2018 presidential elections.
  Particularly, I expected him to have neither ``pairwise'' nor high
  ``positional'' mandate;
\end{frame}


\begin{frame}
  \frametitle{Data}
  \begin{itemize}
    \item I use a ``representative'' street survey a week before the first round
          of the presidential election. A pairwise comparison of the top 4
          candidates was the only question I analyzed.
    \item \textbf{Haddad (Left) (29.28\%)} - Ciro (Center-Left) (12.47\%) -
          Alckmin (Center-Right) (4.76\%) - \textbf{Bolsonaro (Right) (46.9\%)}
          \item Abstention: 20\%
          \item White/Null: 8.79\%
          \item Others: 7.19\%
  \end{itemize}
\end{frame}

\begin{frame}{Data Preprocessing}
  \begin{itemize}
    \item Not all respondents compared all candidates. I imputed the data with polytomous regressions\footnote{Using the \textbf{\textsf{R}} package
          \(\operatorname{mice}.\)}.
    \item There was a discrepancy between the survey and the result of the first
          round. I transferred while respecting Kemeny's distance, and picked
          the transferrence with minimal euclidean distance to the result.
  \end{itemize}
\end{frame}
\begin{frame}
  \frametitle{Method - Saari's Geometry of Voting }

     Positional voting can be normalized:
    \begin{itemize}
      \item Three candidates: \((1,s,0)\) where \(0 \leq s \leq 1\);
      \item Four candidates: \((1,s_{1},s_{2},0)\),
            where \(0 \leq s_{2} \leq s_{1} \leq 1\).
    \end{itemize}
  \end{frame}

\begin{frame}
  \frametitle{Method - Saari's Outcome Triangle}
\begin{figure}[H] \centering \includegraphics[width=\textwidth]{../images/simpletriangle.png}
 \caption{Saari's outcome simplex}
 \label{fig:saari_nurmi}
\end{figure}
\end{frame}


\begin{frame}
  \frametitle{Profile after imputation and rankings transference}
\begin{figure}[!h] \centering

\includegraphics[width=0.85\textwidth, height = 0.85\textheight]{../images/corrected1_indexes_plot.png}



\end{figure}
\end{frame}

\begin{frame}
  \frametitle{Results I: Borda and Condorcet}

\begin{table}[!h]
\centering
\begin{tabular}{rrrrr} & Alckmin & Bolsonaro & Ciro & Haddad \\\hline Alckmin &
  - & -12.63\% & -16.99\% & 8.27\% \\ Bolsonaro & \textcolor{green}{12.63\%} & -
  & \textcolor{green}{5.48\%} & \textcolor{green}{7.46\%} \\ Ciro & 16.99\% &
  -5.48\% & - & 16.65\% \\ Haddad & \textcolor{red}{-8.27\%} &
  \textcolor{red}{-7.46\%} & \textcolor{red}{-16.65\%} & - \\\hline
          \end{tabular}
          \quad

          \begin{tabular}{rrr} \hline & Borda Score & Standardized Borda Score\\
            \hline
            Alckmin & 7029 & 0.464 \\ Bolsonaro & 7718 & \textbf{0.543} \\ Ciro & \textcolor{green}{7756} & \textbf{0.547}\\
            Haddad & 6867 & 0.446 \\ \hline
\end{tabular}
\end{table}

\end{frame}

\begin{frame}
  \frametitle{Opened Tetrahedron - Four candidates Positional Result}
\begin{figure}[!h] \centering \includegraphics[width=\textwidth]{../images/opened_tetrahedron1.png}
 %\caption{Saari's opened tetrahedron }
 
\end{figure}

\end{frame}

\begin{frame}
  \frametitle{Counterfactual Positional Victories }
\begin{table}
  \begin{tabular}{rrrrr}
    \hline\hline
    \textbf{candidates} & \textbf{Alckmin} & \textbf{Bolsonaro} & \textbf{Ciro} & \textbf{Haddad} \\
    \texttt{String} & \texttt{Float64} & \texttt{Float64} & \texttt{Float64} & \texttt{Float64} \\\hline
    Alckmin & 0.0 & 0.32 & 0.0 & 0.57 \\
    Bolsonaro & 0.68 & 0.0 & 0.47 & 0.98 \\
    Ciro & 1.0 & 0.53 & 0.0 & 0.8 \\
    Haddad & 0.43 & 0.02 & 0.2 & 0.0 \\\hline\hline
  \end{tabular}
\end{table}

\end{frame}

\begin{frame}
  \frametitle{Victory in terms of \(s_1\) and \(s_2\)}
\begin{figure}[!h] \centering \includegraphics[width=\columnwidth]{../images/counterfactual_triangle.png}
\end{figure}


\end{frame}

\begin{frame}
  \frametitle{Alternative Set Stability }

   \begin{figure}[!h] \centering
        \begin{subfigure}[b]{0.45\textwidth} \centering
\includegraphics[width=\textwidth]{../images/cw1_nota.png}

        \end{subfigure} \hfill
        \begin{subfigure}[b]{0.45\textwidth} \centering
\includegraphics[width=\textwidth]{../images/cw1_notb.png}

        \end{subfigure} \vskip\baselineskip
        \begin{subfigure}[b]{0.45\textwidth} \centering
\includegraphics[width=\textwidth]{../images/cw1_notc.png}

        \end{subfigure} \hfill
        \begin{subfigure}[b]{0.45\textwidth} \centering
\includegraphics[width=\textwidth]{../images/cw1_noth.png}
        \end{subfigure}
        \caption[ Positional results when dropping one candidate ] {\small
Positional results after dropping one candidate }
        \label{fig:c1dropping}
    \end{figure}

\end{frame}

\begin{frame}
  \frametitle{Discussion}
  \begin{itemize}
    \item We can't conclude Bolsonaro's victory was an institutional fluke.
          However, there is a conflict between the visions of Condorcet and
          Borda in this case.
    \item They perfectly match had he not ran.
  \end{itemize}
\end{frame}

\begin{frame}
  \frametitle{Conclusion}
  \begin{itemize}
    \item Even though the aggregation procedure boosted Bolsonaro's victory, it was not merely its effect, contrary to established theoretical expectations;
    \item But neither was he an undisputed winner:
    \begin{itemize}
      \item pairwise mandate: \textcolor{green}{\checkmark}
      \item positional mandate: \textcolor{red}{\(\times\)}
    \end{itemize}
  \end{itemize}
\end{frame}

\begin{frame}
  \frametitle{Conclusion}
   Next steps:
    \begin{itemize}
      \item Use other variables in the dataset, particularly in the imputation;
      \item Analyze other moments of the 2018 election;
      \item Simulate coalitional and strategic alternative scenarios;
    \end{itemize}
\end{frame}

\frame[allowframebreaks]{

\tiny\printbibliography
}

  \end{document}


%% \section*{Referências}
%% \begin{frame}[allowframebreaks]{Referências}
%% \printbibliography[heading=none]
%% \end{frame}


%%% Local Variables:
%%% mode: latex
%%% TeX-master: ""
%%% End:
